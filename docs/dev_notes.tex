\documentclass[11pt,a4]{article}
\usepackage{html,url}
\usepackage{rotating}

%usage for including an url address
\newcommand{\htmllink}[1]{\htmladdnormallink{\url{#1}}{#1}}
\newcommand{\maillink}[1]{\htmladdnormallink{\url{#1}}{mailto:#1}}

\include{normal_style}

\hoffset = -0.45true in
\textwidth 6.5true in
\textheight 9.5true in
\oddsidemargin 0.25true in
\evensidemargin 0.25true in
\topmargin -0.3true in
\headsep 0.4true in
\parindent 0em
\parskip 0.5em
\def\half{\frac{1}{2}}

\begin{document}
\pagestyle{headings}
\pagenumbering{arabic}
    
\begin{center}
{\Large \bf Development Notes}
\end{center}

Admin creates worksite and uploads a catalogue file, using the catalogue upload
tool, containing the categories of documents relevent to the worksite. The
catalogue import tool simply passes the catalogue file to the redress service
which then imports the categories and content from the file into the redress
database. The trainer is then run across the newly expanded training set
represented by the catalogue database to recreate the training set. The worksite
admin can then supply a list of hostnames to be harvested. A worksite user can
then search across the metadata for the worksite using the redress service.

Do we need an extra field on the database to indicate the catalogue? When Sakai
users search using the metadata tool, do they just search across the harvested
and catalogue data for their worksite?

create\_database.pl - Imports the supplied catalogue xml file into the redress
database, creates a training set and then harvests the supplied urls. When the
creation has finished, sends an email to the supplied address to signal that
the searcher is now usable. The categories are associated with the catalogue
name (which could be the Sakai worksite name).

search.pl - Displays a list of the categories associated with the supplied
worksite id (key)

\begin{enumerate}
\item User logs into Sakai and enters worksite.
\item User clicks on audio tool, snoop applet starts and transmits the client
machine's IP address to service. Service stores IP address against user
ID.
\item User selects another worksite user that they wish to talk to.
\item Service contacts Narada broker and sets up meeting using previously
captured IP addresses.
\item JSP page configures the audio applet params with the
narada ip address, port and the meeting topic number.
\end{enumerate}

\hline

\end{document}
